\documentclass[a4paper,12pt]{article}

\usepackage[hidelinks]{hyperref}
\usepackage{amsmath}
\usepackage{mathtools}
\usepackage{shorttoc}
\usepackage{cmap}
\usepackage[T2A]{fontenc}
\usepackage[utf8]{inputenc}
\usepackage[english, russian]{babel}
\usepackage{xcolor}
\usepackage{graphicx}
\usepackage{float}
\graphicspath{{./img/}}

\definecolor{linkcolor}{HTML}{000000}
\definecolor{urlcolor}{HTML}{0085FF}
\hypersetup{pdfstartview=FitH,  linkcolor=linkcolor,urlcolor=urlcolor, colorlinks=true}

\DeclarePairedDelimiter{\floor}{\lfloor}{\rfloor}

\renewcommand*\contentsname{Содержание}

\newcommand{\plot}[3]{
    \begin{figure}[H]
        \begin{center}
            \includegraphics[scale=0.6]{#1}
            \caption{#2}
            \label{#3}
        \end{center}
    \end{figure}
}

\begin{document}
    \begin{titlepage}
	\begin{center}
		{\large Санкт-Петербургский политехнический университет\\Петра Великого\\}
	\end{center}
	
	\begin{center}
		{\large Физико-механический иститут}
	\end{center}
	
	
	\begin{center}
		{\large Кафедра «Прикладная математика»}
	\end{center}
	
	\vspace{8em}
	
	\begin{center}
		{\bfseries Отчёт по курсовой работе \\по дисциплине «Анализ данных с интервальной неопределённостью»}
	\end{center}
	
	\vspace{5em}
	
	\begin{flushleft}
		\hspace{16em}Выполнил студент:\\\hspace{16em}Игнатьев Даниил Дмитриевич\\\hspace{16em}группа: 5040102/20201
		
		\vspace{2em}
		
		\hspace{16em}Проверил:\\\hspace{16em}к.ф.-м.н., доцент\\\hspace{16em}Баженов Александр Николаевич
		
	\end{flushleft}
	
	
	\vspace{6em}
	
	
	\begin{center}
		Санкт-Петербург\\2023 г.
	\end{center}	
	
\end{titlepage}
    \newpage

    \tableofcontents
    \listoffigures
    \newpage

    \section{Постановка задачи}
    Необходимо для заданных выборок найти точную линейную регрессию, информационные множества и коридоры решений. Сравнить полученные результаты.
    \section{Теория}
    \subsection{Точечная линейная регрессия}
    \quad Рассматривается задача восстановления зависимости для выборки
    $ (X, \textbf(Y))$, $ X = \{x_i\}_{i=1}^{n}, \textbf{Y} = \{\textbf{y}_i\}_{i=1}^{n} $,
    $ x_i $ - точеный, $ \textbf{y}_i $ - интервальный.
    Пусть искомая модель задана в классе линейных функций

    \begin{equation}
        y = \beta_0 + \beta_1 x
        \label{e:model}
    \end{equation}

    Поставим задачу оптимизацию \ref{e:task} для нахождения точечных оценок
    параметров $ \beta_0, \beta_1 $.

    \begin{equation}
        \begin{gathered}
            \sum_{i = 1}^{m}w_{i} \to \min \\
            \text{mid}\textbf{y}_{i} - w_{i} \cdot \text{rad}\textbf{y}_{i} \leq X\beta \leq \text{mid}\textbf{y}_{i} + w_{i} \cdot \text{rad}\textbf{y}_{i} \\
            w_{i} \geq 0, i = 1, ..., m \\
            w, \beta - ?
        \end{gathered}
        \label{e:task}
    \end{equation}
    
    Задачу \ref{e:task} можно решить методами линейного программирования.

    \subsection{Информационное множество}
    \quad \textsl{Информационным множеством} задачи восстановления зависимости
    будем называть множество значений всех параметров зависимости,
    совместных с данными в каком-то смысле. 

    \textsl{Коридором совместных зависимостей} задачи восстановления зависимости
    называется многозначное множество отображений $ \Upsilon $, сопоставляющее
    каждому значению аргумента $ x $ множество
    
    \begin{equation}
        \Upsilon(x) = \bigcup_{\beta \in \Omega} f(x, \beta)
    \end{equation}
    , где $ \Omega $ - информационное множество, $ x $ - вектор переменных, $ \beta $ - вектор оцениваемых параметров. 

    Информационное множество может быть построено, как пересечение полос, заданных
    
    \begin{equation}
        \underline{\textbf{y}_i} \leq \beta_0 + \beta_1 x_{i1} + ... + \beta_m x_{im} \leq \overline{\textbf{y}_i}
    \end{equation}
    , где $ i = \overline{1, n} \textbf{y}_i \in \textbf{Y}, x_i \in X $, $ X $ - точечная выборка переменных,
    $ \textbf{Y} $ - интервальная выборка откликов.

    \section{Реализация}
    \quad Проект реализован на языке Python v. 3.2.5.
    \href{https://github.com/HellInsider/Intervals_2}{GitHub}.

    \section{Результаты}
    \quad Данные $ S_X $ были взяты из файлов \textsl{data/poly\_i.csv}, \newline
    где $ i \in \{0, 1, ... , 9\} $. \newline
    
    Набор значений $ X $ точечный и одинаков для всех выборок. \newline
    $ X = [-0.5, -0.35714286, -0.21428571, -0.07142857, 0.07142857, 0.21428571, 0.35714286, 0.5]  $.
    Набор значений отклика $ Y $ - интервальный.
    
    Построим линейную регрессию и найдём информационное множество.

    Рассмотрим выборку $ Y $ следующим образом. $ y = [\min_{t \in S_i}{S_i}, \max_{t \in S_i}{S_i}]$, \newline
    $ y_i \in Y_1 $.
    \plot{sampleY1}{Визуальзация выборки $ Y_1 $}{p:sampleY1}
        Индекс Жаккара выборки равен $ JK(Y_1) =  -0.052 $
    
    \plot{sampleY2}{Визуальзация выборки $ Y_2 $}{p:sampleY2}
        Индекс Жаккара выборки равен $ JK(Y_2) = -0.088 $
    
    \plot{sampleY3}{Визуальзация выборки $ Y_3 $}{p:sampleY3}
        Индекс Жаккара выборки равен $ JK(Y_3) = -0.057 $
    
    \plot{sampleY4}{Визуальзация выборки $ Y_4 $}{p:sampleY4}
        Индекс Жаккара выборки равен $ JK(Y_4) = -0.036 $
    
    \plot{sampleY5}{Визуальзация выборки $ Y_5 $}{p:sampleY5}
        Индекс Жаккара выборки равен $ JK(Y_5) = -0.04271 $
    
    \plot{sampleY6}{Визуальзация выборки $ Y_6 $}{p:sampleY6}
        Индекс Жаккара выборки равен $ JK(Y_6) = -0.0381 $
    
    \plot{sampleY7}{Визуальзация выборки $ Y_7 $}{p:sampleY7}
        Индекс Жаккара выборки равен $ JK(Y_7) = -0.0369 $
    
    \plot{sampleY8}{Визуальзация выборки $ Y_8 $}{p:sampleY8}
        Индекс Жаккара выборки равен $ JK(Y_8) = -0.0431 $
    
    \plot{sampleY9}{Визуальзация выборки $ Y_9 $}{p:sampleY9}
        Индекс Жаккара выборки равен $ JK(Y_9) = -0.0537 $
    
    % Индекс Жаккара первой выборки равен $ JK(Y_1) = 0.03001 $ (в этой работе $ JK(X) \in [0, 1] $).

    Построим линейную регрессию, решив задачу \ref{e:task} для выборок.
    \plot{PointRegressionsampleY1}{Точечная линейная регрессия для $ Y_1 $}{p:regressionY1}
    Для выборки $ Y_1 $ были получены оценки: $ \beta_0 = 18.333, \beta_1 =19.467  $.
    Тогда полученная модель имеет вид $ y =  18.333 + 19.467 x $.
    
    \plot{PointRegressionsampleY2}{Точечная линейная регрессия для $ Y_2 $}{p:regressionY2}
        Для выборки $ Y_2 $ были получены оценки: $ \beta_0 = 54.28, \beta_1 = -10.53 $.
    Тогда полученная модель имеет вид $ y =  54.28 - 10.53 x $.
    
    \plot{PointRegressionsampleY3}{Точечная линейная регрессия для $ Y_3 $}{p:regressionY3}
        Для выборки $ Y_3 $ были получены оценки: $ \beta_0 = 92.58, \beta_1 = -38.9 $.
    Тогда полученная модель имеет вид $ y = 92.58 -38.93 x $.
    
    \plot{PointRegressionsampleY4}{Точечная линейная регрессия для $ Y_4 $}{p:regressionY4}    
        Для выборки $ Y_4 $ были получены оценки: $ \beta_0 = 77.60, \beta_1 = 221.90 $.
    Тогда полученная модель имеет вид $ y =  77.60 + 221.90 x $.
    
    \plot{PointRegressionsampleY5}{Точечная линейная регрессия для $ Y_5 $}{p:regressionY5}
        Для выборки $ Y_5 $ были получены оценки: $ \beta_0 = 82.68, \beta_1 = 513.76 $.
    Тогда полученная модель имеет вид $ y =  82.68 + 513.76 x $.
    
    \plot{PointRegressionsampleY6}{Точечная линейная регрессия для $ Y_6 $}{p:regressionY6}
        Для выборки $ Y_6 $ были получены оценки: $ \beta_0 = 217.43, \beta_1 = -33.54 $.
    Тогда полученная модель имеет вид $ y =  217.43 - 33.54x $.
    
    \plot{PointRegressionsampleY7}{Точечная линейная регрессия для $ Y_7 $}{p:regressionY7}
        Для выборки $ Y_7 $ были получены оценки: $ \beta_0 = 140.21, \beta_1 = 391.8 $.
    Тогда полученная модель имеет вид $ y =  140.21 + 391.8 x $.
    
    \plot{PointRegressionsampleY8}{Точечная линейная регрессия для $ Y_8 $}{p:regressionY8}
        Для выборки $ Y_8 $ были получены оценки: $ \beta_0 = 200.545, \beta_1 = 32.89 $.
    Тогда полученная модель имеет вид $ y =  200.545 + 32.89 x $.
    
    \plot{PointRegressionsampleY9}{Точечная линейная регрессия для $ Y_9 $}{p:regressionY9}
        Для выборки $ Y_9 $ были получены оценки: $ \beta_0 = 175.14, \beta_1 = 59.28 $.
    Тогда полученная модель имеет вид $ y =  175.14 + 59.28 x $.
    
   

    Информационные множества для данных выборок построить не получилось в виду несовместимости данных.
    %\plot{InformSetY1}{Информационное множество для $ Y_1 $}{p:informSetY1}

    %На рис. \ref{p:informSetY1} можно заметит, что найденные параметры
    %$ \beta_0, \beta_1 $ решением задачи \ref{e:task} лежат внутри информационного %множества.

    %Построим коридор совместных значений для выборки $ Y_1 $ и информационного %множества \ref{p:informSetY1}
    %и оценим значения выходной переменной $ y $ вне пределов значений входной %переменной $ x $.
    %\plot{InformSetCorridorY1}{Коридор совместных значений для $ Y_1 $}%{p:informSetCorridorY1}

    %На рис. \ref{p:informSetCorridorY1} видно, что построенная точечная регрессия %лежит внутри коридора совместных значений,
    %что согласуется с рис. \ref{p:informSetY1}.

    \section{Обсуждение}
    \quad Из полученных результатов в виде отрицательных индексов Жаккара для всех выборок и больших параметров $ \beta_0 , \beta_1 $ можно сделать выводы о несовместимости данных.

\end{document}
