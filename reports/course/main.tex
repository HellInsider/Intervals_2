\documentclass[a4paper,12pt]{article}

\usepackage[hidelinks]{hyperref}
\usepackage{amsmath}
\usepackage{mathtools}
\usepackage{shorttoc}
\usepackage{cmap}
\usepackage[T2A]{fontenc}
\usepackage[utf8]{inputenc}
\usepackage[english, russian]{babel}
\usepackage{xcolor}
\usepackage{graphicx}
\usepackage{float}
\graphicspath{{./img/}}

\definecolor{linkcolor}{HTML}{000000}
\definecolor{urlcolor}{HTML}{0085FF}
\hypersetup{pdfstartview=FitH,  linkcolor=linkcolor,urlcolor=urlcolor, colorlinks=true}

\DeclarePairedDelimiter{\floor}{\lfloor}{\rfloor}

\renewcommand*\contentsname{Содержание}

\newcommand{\plot}[3]{
    \begin{figure}[H]
        \begin{center}
            \includegraphics[scale=0.6]{#1}
            \caption{#2}
            \label{#3}
        \end{center}
    \end{figure}
}

\begin{document}
    \include{title}
    \newpage

    \tableofcontents
    \listoffigures
    \newpage

    \section{Постановка задачи}
    Необходимо для заданных выборок найти точную линейную регрессию, информационные множества и коридоры решений. Сравнить полученные результаты.
    \section{Теория}
    \subsection{Точечная линейная регрессия}
    \quad Рассматривается задача восстановления зависимости для выборки
    $ (X, \textbf(Y))$, $ X = \{x_i\}_{i=1}^{n}, \textbf{Y} = \{\textbf{y}_i\}_{i=1}^{n} $,
    $ x_i $ - точеный, $ \textbf{y}_i $ - интервальный.
    Пусть искомая модель задана в классе линейных функций

    \begin{equation}
        y = \beta_0 + \beta_1 x
        \label{e:model}
    \end{equation}

    Поставим задачу оптимизацию \ref{e:task} для нахождения точечных оценок
    параметров $ \beta_0, \beta_1 $.

    \begin{equation}
        \begin{gathered}
            \sum_{i = 1}^{m}w_{i} \to \min \\
            \text{mid}\textbf{y}_{i} - w_{i} \cdot \text{rad}\textbf{y}_{i} \leq X\beta \leq \text{mid}\textbf{y}_{i} + w_{i} \cdot \text{rad}\textbf{y}_{i} \\
            w_{i} \geq 0, i = 1, ..., m \\
            w, \beta - ?
        \end{gathered}
        \label{e:task}
    \end{equation}
    
    Задачу \ref{e:task} можно решить методами линейного программирования.

    \subsection{Информационное множество}
    \quad \textsl{Информационным множеством} задачи восстановления зависимости
    будем называть множество значений всех параметров зависимости,
    совместных с данными в каком-то смысле. 

    \textsl{Коридором совместных зависимостей} задачи восстановления зависимости
    называется многозначное множество отображений $ \Upsilon $, сопоставляющее
    каждому значению аргумента $ x $ множество
    
    \begin{equation}
        \Upsilon(x) = \bigcup_{\beta \in \Omega} f(x, \beta)
    \end{equation}
    , где $ \Omega $ - информационное множество, $ x $ - вектор переменных, $ \beta $ - вектор оцениваемых параметров. 

    Информационное множество может быть построено, как пересечение полос, заданных
    
    \begin{equation}
        \underline{\textbf{y}_i} \leq \beta_0 + \beta_1 x_{i1} + ... + \beta_m x_{im} \leq \overline{\textbf{y}_i}
    \end{equation}
    , где $ i = \overline{1, n} \textbf{y}_i \in \textbf{Y}, x_i \in X $, $ X $ - точечная выборка переменных,
    $ \textbf{Y} $ - интервальная выборка откликов.

    \section{Реализация}
    \quad Проект реализован на языке Python v. 3.2.5.
    \href{https://github.com/HellInsider/Intervals_2}{GitHub}.

    \section{Результаты}
    \quad Данные $ S_X $ были взяты из файлов \textsl{data/poly\_i.csv}, \newline
    где $ i \in \{0, 1, ... , 9\} $. \newline
    
    Набор значений $ X $ точечный и одинаков для всех выборок. \newline
    $ X = [-0.5, -0.35714286, -0.21428571, -0.07142857, 0.07142857, 0.21428571, 0.35714286, 0.5]  $.
    Набор значений отклика $ Y $ - интервальный.
    
    Построим линейную регрессию и найдём информационное множество.

    Рассмотрим выборку $ Y $ следующим образом. $ y = [\min_{t \in S_i}{S_i}, \max_{t \in S_i}{S_i}]$, \newline
    $ y_i \in Y_1 $.
    \plot{Y1}{Визуальзация выборки, $ Y_1 $}{p:sampleY1}

    % Индекс Жаккара первой выборки равен $ JK(Y_1) = 0.03001 $ (в этой работе $ JK(X) \in [0, 1] $).

    Построим линейную регрессию, решив задачу \ref{e:task} для выборки $ Y_1 $.
    \plot{PointRegressionY1}{Точечная линейная регрессия для $ Y_1 $}{p:regressionY1}
    
    Получим следующие оценки для параметров: $ \beta_0 = 6.333, \beta_1 = 10.267 $.
    Тогда полученная модель имеет вид $ y = 6.333 + 10.267 x $.

    Найдём для данной выборки информационное множество.
    \plot{InformSetY1}{Информационное множество для $ Y_1 $}{p:informSetY1}

    На рис. \ref{p:informSetY1} можно заметит, что найденные параметры
    $ \beta_0, \beta_1 $ решением задачи \ref{e:task} лежат внутри информационного множества.

    Построим коридор совместных значений для выборки $ Y_1 $ и информационного множества \ref{p:informSetY1}
    и оценим значения выходной переменной $ y $ вне пределов значений входной переменной $ x $.
    \plot{InformSetCorridorY1}{Коридор совместных значений для $ Y_1 $}{p:informSetCorridorY1}

    На рис. \ref{p:informSetCorridorY1} видно, что построенная точечная регрессия лежит внутри коридора совместных значений,
    что согласуется с рис. \ref{p:informSetY1}.

    \section{Обсуждение}
    \quad Полученные результаты в виде больших параметров $ \beta_0 , \beta_1 $, а также широкого коридора совместных решений обусловлены большой нтервальной неопределённостью.
    Также стоит отметить, что точечная линейная регрессия попала в информационное множество.
   
\end{document}
